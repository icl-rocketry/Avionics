The Arduino \mbox{\hyperlink{class_sd_fat}{Sd\+Fat}} library provides read/write access to FAT16/\+FAT32 file systems on SD/\+SDHC flash cards.

\mbox{\hyperlink{class_sd_fat}{Sd\+Fat}} requires Arduino 1.\+6x or greater.

Key changes\+:

Support for multiple SPI ports now uses a pointer to a SPIClass object.

See the STM32\+Test example. 
\begin{DoxyCode}{0}
\DoxyCodeLine{explicit SdFat(SPIClass* spiPort);}
\DoxyCodeLine{\(\backslash\)\(\backslash\) or}
\DoxyCodeLine{explicit SdFatEX(SPIClass* spiPort);}

\end{DoxyCode}


Open flags now follow POSIX conventions. O\+\_\+\+RDONLY is the same as O\+\_\+\+READ and O\+\_\+\+WRONLY is the same as O\+\_\+\+WRITE. One and only one of of the mode flags, O\+\_\+\+RDONLY, O\+\_\+\+RDWR, or O\+\_\+\+WRONLY is required.

Open flags for Particle Gen3 and ARM boards are now defined by fcntl.\+h. See \mbox{\hyperlink{_sd_fat_config_8h}{Sd\+Fat\+Config.\+h}} for options.

Support for particle Gen3 devices.

New cluster allocation algorithm.

Please read changes.\+txt and the html documentation in the extras folder for more information.

Please report problems as issues.

A number of configuration options can be set by editing \mbox{\hyperlink{_sd_fat_config_8h}{Sd\+Fat\+Config.\+h}} define macros. See the html documentation for details

Please read the html documentation for this library. Start with html/index.\+html and read the Main Page. Next go to the Classes tab and read the documentation for the classes \mbox{\hyperlink{class_sd_fat}{Sd\+Fat}}, Sd\+Fat\+EX, \mbox{\hyperlink{class_sd_base_file}{Sd\+Base\+File}}, Sd\+File, File, \mbox{\hyperlink{class_stdio_stream}{Stdio\+Stream}}, ifstream, ofstream, and others.

Please continue by reading the html documentation in Sd\+Fat/extras/html

Updated 28 Dec 2018 